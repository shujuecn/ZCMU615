% !TEX program = xelatex
\documentclass[a4paper,11pt]{ctexart}

\usepackage{calc,ifthen}
\usepackage{enumitem}
\usepackage[margin=1.25in]{geometry}

\usepackage{fancyhdr}
\renewcommand{\headrulewidth}{0pt}
\pagestyle{fancy}
\lfoot{}

\setCJKmainfont{方正书宋_GBK}

% \CTEXsetup[format={\Large\bfseries}]{section}
\ctexset{section = {format = {\Large\bfseries}}}

\newlength{\choicelengtha}
\newlength{\choicelengthb}
\newlength{\choicelengthc}
\newlength{\choicelengthd}
\newlength{\choicelengthe}
\newlength{\maxlength}

\makeatletter
\newcommand{\fourch}[4]{
  \par
  \settowidth{\choicelengtha}{A.#1}
  \settowidth{\choicelengthb}{B.#2}
  \settowidth{\choicelengthc}{C.#3}
  \settowidth{\choicelengthd}{D.#4}
  \ifthenelse{\lengthtest{\choicelengtha>\choicelengthb}}{\setlength{\maxlength}{\choicelengtha}}{\setlength{\maxlength}{\choicelengthb}}
  \ifthenelse{\lengthtest{\choicelengthc>\maxlength}}{\setlength{\maxlength}{\choicelengthc}}{}
  \ifthenelse{\lengthtest{\choicelengthd>\maxlength}}{\setlength{\maxlength}{\choicelengthd}}{}
  \ifthenelse{\lengthtest{\maxlength>0.8\linewidth}}
  {%
    \noindent%
    \begin{tabular}{@{}p{\linewidth}@{}}
      \setlength\tabcolsep{0pt}
      \@hangfrom{\textsf A.}#1 \\
      \@hangfrom{\textsf B.}#2 \\
      \@hangfrom{\textsf C.}#3 \\
      \@hangfrom{\textsf D.}#4 \\
    \end{tabular}
  }%
  {%
    \ifthenelse{\lengthtest{\maxlength>0.22\linewidth}}
    {%
      \noindent%
      \begin{tabular}{@{}p{0.48\linewidth}@{\hspace*{0.04\linewidth}}p{0.48\linewidth}@{}}
        \setlength\tabcolsep{0pt}
        \@hangfrom{\textsf A.}#1 & \@hangfrom{\textsf B.}#2 \\
        \@hangfrom{\textsf C.}#3 & \@hangfrom{\textsf D.}#4 \\
      \end{tabular}
    }%
    {%
      \noindent%
      \begin{tabular}{@{}*{3}{p{0.22\linewidth}@{\hspace*{0.04\linewidth}}}p{0.22\linewidth}@{}}
        \setlength\tabcolsep{0pt}
        \@hangfrom{\textsf A.}#1  & \@hangfrom{\textsf B.}#2 & \@hangfrom{\textsf C.}#3 & \@hangfrom{\textsf D.}#4 \\
      \end{tabular}
    }%
  }%
  \unskip\unskip
}

\newcommand{\fivech}[5]{
  \par
  \settowidth{\choicelengtha}{A.#1}
  \settowidth{\choicelengthb}{B.#2}
  \settowidth{\choicelengthc}{C.#3}
  \settowidth{\choicelengthd}{D.#4}
  \settowidth{\choicelengthe}{E.#5}
  \ifthenelse{\lengthtest{\choicelengtha>\choicelengthb}}{\setlength{\maxlength}{\choicelengtha}}{\setlength{\maxlength}{\choicelengthb}}
  \ifthenelse{\lengthtest{\choicelengthc>\maxlength}}{\setlength{\maxlength}{\choicelengthc}}{}
  \ifthenelse{\lengthtest{\choicelengthd>\maxlength}}{\setlength{\maxlength}{\choicelengthd}}{}
  \ifthenelse{\lengthtest{\choicelengthe>\maxlength}}{\setlength{\maxlength}{\choicelengthe}}{}
  \ifthenelse{\lengthtest{\maxlength>0.5\linewidth}}
  {%
    \noindent%
    \begin{tabular}{@{}p{\linewidth}@{}}
      \setlength\tabcolsep{0pt}
      \@hangfrom{\textsf A.}#1 \\
      \@hangfrom{\textsf B.}#2 \\
      \@hangfrom{\textsf C.}#3 \\
      \@hangfrom{\textsf D.}#4 \\
      \@hangfrom{\textsf E.}#5 \\
    \end{tabular}
  }%
  {%
    \ifthenelse{\lengthtest{\maxlength>0.15\linewidth}}
    {%
      \noindent%
      \begin{tabular}{@{}p{0.48\linewidth}@{\hspace*{0.04\linewidth}}p{0.48\linewidth}@{}}
        \setlength\tabcolsep{0pt}
        \@hangfrom{\textsf A.}#1 & \@hangfrom{\textsf B.}#2 \\
        \@hangfrom{\textsf C.}#3 & \@hangfrom{\textsf D.}#4 \\
        \@hangfrom{\textsf E.}#5 \\
      \end{tabular}
    }%
    {%
      \noindent%
      \begin{tabular}{@{}*{5}{p{0.16\linewidth}@{\hspace*{0.04\linewidth}}}p{0.16\linewidth}@{}}
        \setlength\tabcolsep{0pt}
        \@hangfrom{\textsf A.}#1  & \@hangfrom{\textsf B.}#2 & \@hangfrom{\textsf C.}#3 & \@hangfrom{\textsf D.}#4 & \@hangfrom{\textsf E.}#5 \\
      \end{tabular}
    }%
  }%
  \unskip\unskip
}
\makeatother

% \renewcommand{\thesection}{\chinese{section}、}
% \renewcommand{\thesubsection}{\arabic{subsection}.}
% \renewcommand{\thesubsubsection}{(\arabic{subsubsection})}

% \makeatletter
% \renewcommand\@seccntformat[1]{%
%     {\csname the#1\endcsname}\hspace{0.1em}
% }
% \makeatother



\title{全国研究生招生考试 \\ 浙江中医药大学2021年攻读硕士学位研究生入学考试 \\ 《中医综合》试卷A代码615}
\author{开源:https://github.com/shujuecn/ZCMU615}
\date{}

\begin{document}

\maketitle

\section*{\large 一、A型题:1-80小题,每小题1.5分,共120分。在每小题给出的A、B、C、D四个选项中,请选出一项最符合题目要求的。}

\begin{enumerate}
      \item 主张“六气皆从火化”“五志过极皆能生火”的医家是(\quad)
            \fourch{李杲}{朱丹溪}{张从正}{刘完素}
      \item “寒者热之”的治法是阴阳哪一关系的具体应用(\quad)
            \fourch{对立制约}{互根互用}{消长平衡}{相互转化}
      \item “泻南补北法”适用于(\quad)
            \fourch{肝火犯肺证}{心肾不交证}{肝脾不和证}{脾虚水泛证}
      \item 称为“罢极之本”的是(\quad)
            \fourch{肺}{心}{肝}{肾}
      \item 肾的主要功能是(\quad)
            \fourch{主纳气}{主行水}{主疏泄}{主运化水液}
      \item 人体生命活动的原动力是(\quad)
            \fourch{营气}{卫气}{宗气}{元气}
      \item “吐下之余,定无完气”是因为(\quad)
            \fourch{津能载气}{津能生气}{气能摄津}{气能行津}
      \item 脾在体为(\quad)
            \fourch{肉}{筋}{脉}{骨}
      \item 主司眼睑开合的是(\quad)
            \fourch{督脉}{跷脉}{任脉}{冲脉}
      \item 情志致病,影响脏腑气机,恐则(\quad)
            \fourch{气结}{气逆}{气缓}{气下}
      \item 寒邪的致病特点是(\quad)
            \fourch{伤上}{凝滞}{开泄}{善行}
      \item 真实假虚证的治疗原则应是(\quad)
            \fourch{祛邪兼扶正}{扶正兼祛邪}{先祛邪后扶正}{单独祛邪}
      \item 属于五行相克关系失常所致的病变是(\quad)
            \fourch{肾病及肝}{肝病及脾}{肝病及心}{肺病及肾}
      \item “用寒远寒,用热远热”属于(\quad)
            \fourch{因地制宜}{因人制宜}{因时制宜}{因材制宜}
      \item 下列各项,属于“症状”的是(\quad)
            \fourch{心烦失眠}{舌苔薄黄}{腹如舟状}{脉细无力}
      \item 脏躁可表现为(\quad)
            \fourch{淡漠痴呆}{猝然昏仆}{焦虑恐惧}{狂躁不安}
      \item 面色与口唇青紫者,多属(\quad)
            \fourch{阴寒内盛}{肝郁脾虚}{心血瘀阻}{高热抽搐}
      \item 面部脏腑分候中,左颊候(\quad)
            \fourch{心}{肝}{脾}{肺}
      \item 下列各项,属于是形盛气虚的表现的是(\quad)
            \fourch{体胖能食,肌肉坚实}{体胖食少,神疲乏力}{形瘦能食,舌红苔黄}{形瘦颧红,皮肤干焦}
      \item 小儿凶门凸出,多属于(\quad)
            \fourch{温病火邪上攻,脑髓有病}{吐泻伤津,或气血不足}{肾精亏虚,脑髓失充}{肾气不足,发育不良}
      \item 睡中咬牙啮齿,多辨证为(\quad)
            \fourch{肾阴虚}{胃热或虫积}{肝风内动}{阳明热盛}
      \item 患者全目赤肿,伴见发热、微恶风寒,巅顶头痛,舌红,苔薄黄,脉浮数者,最宜诊断为(\quad)
            \fourch{肝胆湿热}{肝经风热}{肝郁气滞}{肝火炽盛}
      \item 鉴别表证与里证主要根据(\quad)
            \fourch{寒热是否并见}{头痛轻重}{腹痛喜按拒按}{咳嗽有无黄痰}
      \item 谵语的病机是(\quad)
            \fourch{心气大伤}{热扰心神}{心气不足}{痰蒙心神}
      \item 湿温潮热的特点是(\quad)
            \fourch{午后低热}{日哺热 甚}{身热不扬}{骨蒸发热}
      \item 头痛连项者,属(\quad)
            \fourch{太阳经头痛}{少阳经头痛}{阳明经头痛}{厥阴经头痛}
      \item 厌食油腻厚味,胁肋胀痛,苔黄腻者,多为(\quad)
            \fourch{伤食}{气滞}{肝胆病}{脾胃病}
      \item 脉“有根”主要是指(\quad)
            \fourch{不浮不沉,不大不小}{柔和有力,节律一致}{不快不慢,和缓有力}{尺脉有力,沉取不绝}
      \item 延胡索醋制的目的是(\quad)
            \fourch{改变药性}{减低毒性}{增强疗效}{矫味矫臭}
      \item 下列功效术语中,属于对病功效的是(\quad)
            \fourch{大补元气}{止痛}{调和营卫}{截疟}
      \item 黄连配木香治疗湿热泻痢属于下列哪种配伍关系(\quad)
            \fourch{单行}{相使}{相反}{相恶}
      \item 入汤剂宜后下的药物是(\quad)
            \fourch{钩藤、牡蛎}{车前子、番泻叶}{西洋参、羚羊角}{肉桂、薄荷}
      \item 既能解表散寒,又擅温肺化饮的药物是(\quad)
            \fourch{生姜}{羌活}{细辛}{香薷}
      \item 其性甘寒清润,既能“专清阴中之热”,又擅“疗肺热有余咳嗽”之药的是(\quad)
            \fourch{青蒿}{地骨皮}{银柴胡}{白薇}
      \item 善疗风湿,其“驱逐寒湿之力甚捷”,又“必须沉寒痼冷,足以相当”之药是(\quad)
            \fourch{羌活}{川乌}{蕲蛇}{木瓜}
      \item 其性苦、辛,微寒,能“利水道而泻湿淫,消瘀热而退黄疸”,“为治湿热黄疸之要药”的是(\quad)
            \fourch{茵陈}{金钱草}{虎杖}{车前草}
      \item 善“散厥阴之寒”,“疏肝气有偏长”,为治寒凝肝脉诸痛之要药的是(\quad)
            \fourch{柴胡}{川楝子}{青皮}{吴茱萸}
      \item 善“消肉食之积”,凡“伤诸肉者,必用之药”,为治油腻肉食积滞之要药的是(\quad)
            \fourch{山楂}{神曲}{莱菔子}{鸡内金}
      \item 其性苦,温。“用之以补接伤碎最神”,为伤科要药的是(\quad)
            \fourch{骨碎补}{延胡索}{乳香}{姜黄}
      \item 其性辛温,凡有痰之处无不尽消,“痰在皮里膜外,非此不达”之品是(\quad)
            \fourch{白附子}{天南星}{白前}{芥子}
      \item 为交通心肾,安定神志,益智强识之佳品,又能祛痰、消肿的是(\quad)
            \fourch{半夏}{石菖蒲}{远志}{琥珀}
      \item 味酸甘且涩,性平,“其功全在固涩”,既具固精缩尿之功,又有固崩止带之能,还具涩肠止泻之效的是(\quad)
            \fourch{海螵蛸}{金樱子}{芡实}{莲子}
      \item 最早按病证分类的方书是(\quad)
            \fourch{《太平圣惠方》}{《备急千金要方》}{《五十二病方》}{《太平惠民和剂局方》}
      \item 患者但咳,身热不甚,口微渴,脉浮数,治当首选的方剂是(\quad)
            \fourch{银翘散}{止嗽散}{桑菊饮}{杏苏散}
      \item 体现“透热转气”治法的方剂是(\quad)
            \fourch{犀角地黄汤}{清营汤}{竹叶石膏汤}{青蒿鳖甲汤}
      \item 香薷散的功用是(\quad)
            \fourch{解表散寒,化湿和中}{祛暑解表,化湿和中}{祛暑解表,清热化湿}{解表散寒,温肺化饮}
      \item 既能治疗脾虚湿盛之泄泻,又能治疗肺脾气虚,痰湿咳嗽的方剂是(\quad)
            \fourch{四君子汤}{参苓白术散}{二陈汤}{小青龙汤}
      \item 牡蛎散和玉屏散组成中均含有的药物是(\quad)
            \fourch{麻黄根}{防风}{白术}{黄芪}
      \item 酸枣仁汤中疏肝气,配合君药养血调肝的药物是(\quad)
            \fourch{香附}{柴胡}{川芎}{川楝子}
      \item 越鞠丸方中针对湿郁而设的药物是(\quad)
            \fourch{香附}{苍术}{神曲}{栀子}
      \item 重用当归为君药的方剂是(\quad)
            \fourch{温经汤}{四物汤}{生化汤}{当归补血汤}
      \item 治疗肺胃阴伤,火逆上气之虚热肺痿,最宜选用的方剂是(\quad)
            \fourch{泻白散}{麦门冬汤}{炙甘草汤}{养阴清肺汤}
      \item 病人症见院腹胀满,不思饮食,口淡无味,恶心呕吐,嗳气吞酸,肢体沉重,怠惰嗜卧,舌苔白腻而厚,脉缓。治宜首选(\quad)
            \fourch{平胃散}{保和丸}{参苓白术散}{藿香正气散}
      \item 苓甘五味姜辛汤的君药是(\quad)
            \fourch{茯苓}{干姜}{五味子}{细辛}
      \item 保和丸中配伍连翘的意义是(\quad)
            \fourch{清热解毒}{清热散结}{散结消肿}{疏散邪热}
      \item 乌梅丸主治蛔厥证的病机是(\quad)
            \fourch{中焦虚寒,蛔虫上扰}{肠寒胃热,蛔虫上扰}{肝胃热盛,蛔虫上扰}{肝肾虚寒,蛔虫上扰}
      \item 将温病分为新感与伏气两大类进行辨证施治的医家是(\quad)
            \fourch{吴又可}{吴鞠通}{薛生白}{王孟英}
      \item 患者恶寒重,发热轻,四肢欠温,语音低微,舌质淡胖,脉沉细无力,治宜选用(\quad)
            \fourch{参苏饮}{再造散}{加减葳蕤汤}{荆防败毒散}
      \item 患者咯吐大量脓血痰,异常腥臭,有时咯血,胸中满痛,甚则气喘不能平卧,发热,面赤,烦渴喜饮,舌质红,舌苔黄腻,脉滑数,治宜选用(\quad)
            \fourch{沙参麦冬汤}{千金苇茎汤}{如金解毒散}{加味桔梗汤}
      \item 痰湿蕴肺型咳嗽的主证特点是(\quad)
            \fourch{咳声气促、痰多质粘}{咳声重浊、痰黄量少}{咳声重浊、痰多胸闷}{咳嗽频剧、痰少而粘}
      \item 胸痹心痛病属本虚标实,标实当泻,以何法为重?(\quad)
            \fourch{活血通脉}{辛温通阳}{化痰泄浊}{理气宽胸}
      \item 患者证见心烦不寐,胸闷院痞,伴口苦,头重,嗳气,目眩,舌红苔黄腻,脉滑数。治宜首选(\quad)
            \fourch{龙胆泻肝汤}{黄连温胆汤}{天王补心丹}{交泰丸}
      \item 患者心下痞满,胸膈满闷,烦躁口苦,口干饮水不多,大便秘结,舌质红,苔黄腻,脉滑数,治宜选用(\quad)
            \fourch{清中汤}{大承气汤}{黄连温胆汤}{泻心汤}
      \item 治疗因伤于肉食而致呕吐者,常重用(\quad)
            \fourch{谷芽}{麦芽}{山楂}{神曲}
      \item 呃逆的治疗原则是(\quad)
            \fourch{理气和胃,降逆平呃}{舒肝和胃,降逆止呃}{温补脾胃止呃}{温中散寒,降逆止呕}
      \item 患者证见腹大胀满,按之如囊裹水,脘腹痞胀,得热则舒,小便短少,大便薄,舌苔白腻,脉弦迟,治宜选用(\quad)
            \fourch{中满分消饮}{附子理苓汤}{胃苓汤}{实脾饮}
      \item 积聚气滞血阻证,其治法是(\quad)
            \fourch{补益气血,化瘀消积}{理气活血,消积散瘀}{理气化瘀,导气通腑}{疏肝解郁,行气消聚}
      \item “癃闭”之名,首见于(\quad)
            \fourch{《济生方》}{《内经》}{《医学入门》}{《诸病源候论方》}
      \item 水湿浸渍之水肿,如湿郁化热,但湿重于热,治宜选用(\quad)
            \fourch{五皮饮合茯苓汤}{越婢加术汤}{三仁汤合五皮饮}{疏凿饮子}
      \item 按痰饮停积的部位分类,饮留胃肠的是(\quad)
            \fourch{痰饮}{支饮}{溢饮}{悬饮}
      \item 虚劳病的基本病机为(\quad)
            \fourch{气血阴阳失调}{气血阴阳亏虚,脏腑功能失调}{气血阴阳亏虚,脏腑虚损}{阴阳失调,营卫不和}
      \item 患者间断发作肉眼血尿,伴有头晕耳鸣,五心烦热,腰膝痰软,舌红,苔薄黄,脉细数。治宜选用(\quad)
            \fourch{小蓟饮子}{导赤散}{知柏地黄丸}{茜根散}
      \item 肥胖的治疗原则是(\quad)
            \fourch{补虚泻实}{健脾益气}{调和气血}{化痰散瘀}
      \item 按照十二经脉的气血流注顺序,下列正确的是(\quad)
            \fourch{手少阴经 - 手太阳经 - 足少阴经}{足阳明经 - 足太阴经 - 手少阴经}{足太阴经 - 手厥阴经 - 手少阳经}{足太阳经 - 手厥阴经 - 手少阳经}
      \item 小指末节桡侧,距指甲角0.1寸的穴是(\quad)
            \fourch{少泽}{少商}{中冲}{少冲}
      \item 治疗功能性子宫出血、阳痿、遗精,头针首选(\quad)
            \fourch{额中线}{额旁1线}{额旁2线}{额旁3线}
      \item 阳经五输穴按照井荥输经合的顺序,其五行属性正确的是(\quad)
            \fourch{木火土金水}{金水木火土}{金木水火土}{木火金水上}
      \item 下列腧穴中不属于背俞穴的是(\quad)
            \fourch{肺俞}{心俞}{膈俞}{厥阴俞}
      \item 治疗胆绞痛急性发作,电针波型应首选(\quad)
            \fourch{疏波}{疏密波}{密波}{断续波}
      \item 根据骨度折量定位法,前发际正中至后发际正中是(\quad)
            \fourch{8寸}{9寸}{12寸}{16寸}
\end{enumerate}

\section*{\large 二、B型题:81-120小题:每小题1.5分,共60分。A、B、C、D是其下两道小题的备选项,请从中选择一项最符合题目要求的。每个选项可以被选择一次或两次。}

\begin{enumerate}[resume]
      \item[]
            \fourch{主疏泄}{朝百脉}{主运化}{主血脉}
      \item 肺的功能是(\quad)
      \item 心的功能是(\quad)
            \fourch{卫气}{营气}{宗气}{元气}
      \item 参与血液生成的气是(\quad)
      \item 具有调控腠理开合作用的气是(\quad)
            \fourch{卒发}{继发}{复发}{伏法}
      \item 小儿食积日久致“疳积”属于(\quad)
      \item 误食毒物致人迅速发病属于(\quad)
            \fourch{形体羸弱,精神萎靡}{目似有光,颧红如妆}{神昏谵语,循衣摸床}{精神不振,面色少华}
      \item 属于邪盛神乱而失神是(\quad)
      \item 少神的表现是(\quad)
            \fourch{中焦火炽}{脏腑精气衰竭}{气血亏虚}{阴虚内热}
      \item 若消瘦伴五心烦热、潮热盗汗,则为(\quad)
      \item 若久病卧床不起,骨瘦如柴者,则为(\quad)
            \fourch{精血不足}{血虚受风}{疳积病}{肾虚或血热}
      \item 患者发黄干枯,稀疏易落,多辨证为(\quad)
      \item 患者突然片状脱发,显露圆形或椭圆形光亮头皮而无自觉症状者,多为(\quad)
            \fourch{黄芪}{当归}{天麻}{神曲}
      \item 道地药材产于福建的是(\quad)
      \item 道地药材产于贵州的是(\quad)
            \fourch{甘,寒}{苦,寒}{辛,热}{辛、苦,温}
      \item 茜草的性味为(\quad)
      \item 艾叶的性味为(\quad)
            \fourch{苦杏仁}{枇杷叶}{紫菀}{款冬花}
      \item 其药性辛、苦,温。功能润肺下气,化痰止咳的是(\quad)
      \item 其药性辛、微苦,温。功能润肺下气,止咳化痰的是(\quad)
            \fourch{寒热并调,和胃降逆}{和胃补中,降逆消痞}{和胃消痞,宣散水气}{寒热平调,散结除痞}
      \item 半夏泻心汤的功用是(\quad)
      \item 生姜泻心汤的功用是(\quad)
            \fourch{白头翁汤}{槐花散}{黄土汤}{四物汤}
      \item 脾阳不足,脾不统血之大便出血,可选用(\quad)
      \item 肠风脏毒之大便出血,可选用(\quad)
            \fourch{生地黄、栀子、黄芩}{麻黄、杏仁、甘草}{枳壳、芍药、香附}{生地黄、玄参、麦冬}
      \item 龙胆泻肝汤中含有(\quad)
      \item 柴胡疏肝散中含有(\quad)
            \fourch{麻杏石甘汤}{月华丸}{清金化痰汤}{沙参麦冬汤}
      \item 咳嗽肺阴亏耗证,治宜选用(\quad)
      \item 肺痨肺阴亏损证,治宜选用(\quad)
            \fourch{龙胆泻肝汤}{柴胡疏肝散}{天麻钩藤饮}{半夏白术天麻汤}
      \item 眩晕肝阳上亢证,治宜选用(\quad)
      \item 不寐肝火扰心证,治宜选用(\quad)
            \fourch{髓减脑消,神机失用}{阴阳失调,气血逆乱}{气机逆乱,气血阴阳不相顺接}{风火痰瘀扰乱,清窍失宁}
      \item 中风的主要病机是(\quad)
      \item 厥证的主要病机是(\quad)
            \fourch{屡攻屡补,以平为期}{治实当顺虚,补虚勿忘实}{结者散之,留者攻之}{大积大聚,其可犯者,衰其大半而止}
      \item 有关积聚的治疗,《素问·六元正纪大论》提出(\quad)
      \item 有关积聚的治疗,《医宗必读》提出(\quad)
            \fourch{手阳明大肠经}{手太阳小肠经}{足阳明胃经}{任脉}
      \item 大肠的募穴位于(\quad)
      \item 小肠的募穴位于(\quad)
            \fourch{督脉}{任脉}{冲脉}{带脉}
      \item 足临泣为八脉交会穴,通(\quad)
      \item 公孙为八脉交会穴,通(\quad)
            \fourch{舒张进针法}{夹持进针法}{指切进针法}{提捏进针法}
      \item 针刺印堂穴宜首选(\quad)
      \item 针刺环跳穴宜首选(\quad)
            \fourch{中脘}{建里}{关元}{中极}
      \item 位于脐上4寸的穴位是(\quad)
      \item 位于脐下4寸的穴位是(\quad)
\end{enumerate}

\section*{\large 三、X型题:第121-180题,每小题2分,共120分。在每小题给出的A、B、C、D四个选项中,至少有两项是符合题目要求的。请选出所有符合题目要求的答案。多选或少选均不得分。}

\begin{enumerate}[resume]
      \item 与宗气盛衰有关的是(\quad)
            \fourch{呼吸}{语言}{发声}{温养}
      \item 影血液运行的因素有(\quad)
            \fourch{气的固摄}{脉道的通畅}{气的推动}{气的温煦}
      \item 湿邪的致病特点是(\quad)
            \fourch{重浊}{黏滞}{趋下}{收引}
      \item 适用“通因通用”治疗的是(\quad)
            \fourch{瘀血所致崩漏}{气虚所致便秘}{阳虚所致癃闭}{食滞所致泄泻}
      \item 偏阳质易发的疾病是(\quad)
            \fourch{失眠}{出血}{水肿}{痰饮}
      \item 五味属于阳的(\quad)
            \fourch{酸}{苦}{辛}{甘}
      \item 肝肾关系可概括为(\quad)
            \fourch{精血同源}{藏泄互用}{阴阳互滋互制}{精神互用}
      \item 参与津液代谢的脏腑是(\quad)
            \fourch{三焦}{肺}{脾}{肾}
      \item 痰饮的致病特点是(\quad)
            \fourch{阻滞气血运行}{易于蒙蔽心神}{症状复杂}{变幻多端}
      \item 下列脏腑中属于奇恒之府的是(\quad)
            \fourch{三焦}{胆}{脑}{女子胞}
      \item 点、刺舌多见于(\quad)
            \fourch{脾肾阳虚}{胃肠热盛}{脏腑热极}{血分热盛}
      \item “癃闭”形成的原因有(\quad)
            \fourch{湿热下注}{瘀血阻滞}{砂石阻塞}{肾阳不足}
      \item 寒证的典型表现有(\quad)
            \fourch{恶寒喜暖}{脘腹冷痛}{咳嗽痰稀白}{面色淡白}
      \item 气滞证的疼痛特点,可包括下列哪几项(\quad)
            \fourch{按之一般有形}{部位多不固定}{随情绪而增减}{症状时轻时重}
      \item 下列各项中,属于血瘀证的色脉改变的有(\quad)
            \fourch{皮肤有紫色斑块}{腹壁青筋暴露}{出血色紫黯夹块}{舌淡胖苔腻或滑}
      \item 阳水与阴水的鉴别点有哪些(\quad)
            \fourch{发病缓急}{小便量多少}{病程的长短}{上肢肿甚或下肢肿甚}
      \item 肝血虚证可见(\quad)
            \fourch{夜盲}{关节拘急}{两颧潮红}{月经色淡}
      \item 下列各项,属于气虚证、脾阳虚证、脾虚气陷证、脾不统血证共同表现的是(\quad)
            \fourch{腹胀纳少}{便溏肢倦}{食少懒言}{腹痛喜按}
      \item 肾精不足的临床表现有(\quad)
            \fourch{健忘恍惚}{囟门迟闭}{足软无力}{齿松发脱}
      \item 下列哪些属六经病中“传经”(\quad)
            \fourch{逆经传}{表里传}{循经传}{越经传}
      \item 下列属于《本草经集注》学术成就的是(\quad)
            \fourch{初步构建了综合性本草的编写模式}{首创按自然属性分类法}{开创了以病类药之先河}{开创了图文并茂的药学著作编撰之先例}
      \item 下列药物中,属于当代中药配伍禁忌,与半夏相反的是(\quad)
            \fourch{川乌}{草乌}{附子}{乌药}
      \item 下列属于桂枝适应症的是(\quad)
            \fourch{风寒表证}{风寒湿痹,肩臂疼痛}{脾阳不运,水湿内停之痰饮眩晕}{肾阳不足,膀胱气化不行之水肿}
      \item 下列中药具有清热燥湿功效的是(\quad)
            \fourch{鱼腥草}{黄芩}{龙胆}{栀子}
      \item 下列药物中具有活血化瘀功效的是(\quad)
            \fourch{大黄}{牡丹皮}{当归}{三七}
      \item 下列属于附子适应症的有(\quad)
            \fourch{亡阳兼气脱证}{素体阳虚,复感外寒}{脾胃虚寒证}{风寒湿痹}
      \item 下列属于川芎适应症的有(\quad)
            \fourch{血瘀经闭、痛经}{阴虚阳亢头痛}{风湿痹痛}{胸中瘀血,胸胁刺痛}
      \item 下列属于牡蛎适应症的有(\quad)
            \fourch{肝阳上亢证}{自汗、盗汗}{瘀血证}{瘿瘤瘰疬}
      \item 下列药物中善补脾肺之气的药物有(\quad)
            \fourch{黄芪}{白术}{黄精}{五味子}
      \item 下列药物中,善于养胃阴的药物有(\quad)
            \fourch{阿胶}{天冬}{石斛}{玉竹}
      \item 方剂运用的变化形式包括(\quad)
            \fourch{药量增减的变化}{药味加减的变化}{剂型更换的变化}{服法调整的变化}
      \item 柴胡在四逆散中的配伍特点意义是(\quad)
            \fourch{透邪外出}{升发阳气}{和解少阳}{琉肝解郁}
      \item 能体现“通因通用”治法的方剂是(\quad)
            \fourch{大承气汤}{黄龙汤}{桂枝茯苓丸}{芍药汤}
      \item 四君子汤、补中益气汤两方共有的药物是(\quad)
            \fourch{黄芪}{人参}{白术}{茯苓}
      \item 半夏、生姜同用的方剂是(\quad)
            \fourch{苏子降气汤}{旋覆代赭汤}{枳实消痞丸}{丁香柿蒂汤}
      \item 组成中均含小蓟、栀子的方剂是(\quad)
            \fourch{十灰散}{咳血方}{温经汤}{小蓟饮子}
      \item 仙方活命饮的功用是(\quad)
            \fourch{活血止痛}{消肿溃坚}{逐瘀排脓}{清热解毒}
      \item 独活寄生汤的主治证包括(\quad)
            \fourch{腰膝疼痛}{屈伸不利}{畏寒喜暖}{麻木不仁}
      \item 温胆汤主治证的临床表现可见(\quad)
            \fourch{胆怯易惊}{虚烦不宁}{心下痞闷}{眩晕呕恶}
      \item 阳和汤可以治疗的病证有(\quad)
            \fourch{贴骨疽}{鹤膝风}{脱疽}{痰核}
      \item 下列属于热哮主症的是(\quad)
            \fourch{痰鸣如吼}{痰粘色黄}{口干口苦}{舌苔白腻,脉濡滑}
      \item 下列哪些病症迁延日久,可转化成肺痿(\quad)
            \fourch{咳嗽}{肺痈}{肺痨}{喘证}
      \item 下列关于腹痛的叙述,正确的是(\quad)
            \fourch{内科腹痛常先腹痛后发热}{外科腹痛常先发热后腹痛}{外科腹痛常痛有定处,压痛明显}{内科腹痛一般痛无定处,压痛不明显}
      \item 反胃的治疗原则是(\quad)
            \fourch{润燥生津}{降逆和胃}{温补脾肾}{温中健牌}
      \item 下列哪项不是癫狂的主要病因(\quad)
            \fourch{七情内伤}{先天不足}{跌扑外伤}{外邪侵袭}
      \item 少阳头痛可选用的药物是(\quad)
            \fourch{柴胡}{黄芩}{白芷}{川芎}
      \item 淋证的发生,主要与下列哪些脏器有关(\quad)
            \fourch{肝}{脾}{肾}{膀胱}
      \item 《先醒斋医学广笔记·吐血》认为治疗吐血(\quad)
            \fourch{宜行气不宜破气}{宜降气不宜降火}{宜行血不宜止血}{宜补肝不宜伐肝}
      \item 患者肌肉、关节刺痛数年,痛处固定不移,关节局部肿胀,按之稍硬,屈伸不利,舌质紫暗,苔白腻,脉弦涩。其治法是(\quad)
            \fourch{祛风散寒}{除湿通络}{蠲痹通络}{化痰行瘀}
      \item 《丹溪心法》认为腰痛的病因包括(\quad)
            \fourch{湿热}{肾虚}{瘀血}{痰积}
      \item 位于第2腰椎棘突水平的穴位有(\quad)
            \fourch{命门}{肾俞}{志室}{腰眼}
      \item 可能引起滞针的原因有(\quad)
            \fourch{患者精神紧张}{医者手法不当}{患者体位改变}{留针时间过长}
      \item 以下穴位中,既是合穴又是下合穴的是(\quad)
            \fourch{曲池}{阳陵泉}{小海}{足三里}
      \item 以下穴位归经描述正确的是(\quad)
            \fourch{太白 - 足太阴脾经}{劳宫 - 手少阴心经}{复溜 - 足少阴肾经}{太冲 - 足厥阴肝经}
      \item 有关十二经脉循行走向规律描述正确的是(\quad)
            \fourch{手三阴从胸走手}{手三阳从手走头}{足三阳从头走足}{足三阴从足走腹胸}
      \item 下列有关经脉腧穴总数描述正确的是(\quad)
            \fourch{足阳明胃经44}{足少阳胆经45}{足太阴脾经21}{手阳明大肠经20}
      \item 有关“透天凉”描述正确的是(\quad)
            \fourch{多用于治疗各种实热性疾病}{操作先浅后深}{每层做紧按慢提六数}{出针时摇大针孔}
      \item 两穴之间相距3寸的是(\quad)
            \fourch{曲池与手三里}{阳溪与偏厉}{郄门与内关}{支沟与阳池}
      \item 以下有关经脉循行描述正确的是(\quad)
            \fourch{手少阴心经起于心中}{手厥阴心包经起于胸中}{手太阴肺经起于肺部}{足阳明胃经起于鼻旁}
      \item 以下穴位中属于手少阳三焦经的是(\quad)
            \fourch{肩髎}{翳风}{肩井}{耳门}

\end{enumerate}

\end{document}
