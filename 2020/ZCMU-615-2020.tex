% !TEX program = xelatex
\documentclass[a4paper,11pt]{ctexart}

\usepackage{calc,ifthen}
\usepackage{enumitem}
\usepackage[margin=1.25in]{geometry}
\usepackage{fancyhdr}
\renewcommand{\headrulewidth}{0pt}
\pagestyle{fancy}
\lfoot{}
\setCJKmainfont{方正书宋_GBK}

% \CTEXsetup[format={\Large\bfseries}]{section}
\ctexset{section = {format = {\Large\bfseries}}}

\newlength{\choicelengtha}
\newlength{\choicelengthb}
\newlength{\choicelengthc}
\newlength{\choicelengthd}
\newlength{\choicelengthe}
\newlength{\maxlength}

\makeatletter
\newcommand{\fourch}[4]{
  \par
  \settowidth{\choicelengtha}{A.#1}
  \settowidth{\choicelengthb}{B.#2}
  \settowidth{\choicelengthc}{C.#3}
  \settowidth{\choicelengthd}{D.#4}
  \ifthenelse{\lengthtest{\choicelengtha>\choicelengthb}}{\setlength{\maxlength}{\choicelengtha}}{\setlength{\maxlength}{\choicelengthb}}
  \ifthenelse{\lengthtest{\choicelengthc>\maxlength}}{\setlength{\maxlength}{\choicelengthc}}{}
  \ifthenelse{\lengthtest{\choicelengthd>\maxlength}}{\setlength{\maxlength}{\choicelengthd}}{}
  \ifthenelse{\lengthtest{\maxlength>0.8\linewidth}}
  {%
    \noindent%
    \begin{tabular}{@{}p{\linewidth}@{}}
      \setlength\tabcolsep{0pt}
      \@hangfrom{\textsf A.}#1 \\
      \@hangfrom{\textsf B.}#2 \\
      \@hangfrom{\textsf C.}#3 \\
      \@hangfrom{\textsf D.}#4 \\
    \end{tabular}
  }%
  {%
    \ifthenelse{\lengthtest{\maxlength>0.22\linewidth}}
    {%
      \noindent%
      \begin{tabular}{@{}p{0.48\linewidth}@{\hspace*{0.04\linewidth}}p{0.48\linewidth}@{}}
        \setlength\tabcolsep{0pt}
        \@hangfrom{\textsf A.}#1 & \@hangfrom{\textsf B.}#2 \\
        \@hangfrom{\textsf C.}#3 & \@hangfrom{\textsf D.}#4 \\
      \end{tabular}
    }%
    {%
      \noindent%
      \begin{tabular}{@{}*{3}{p{0.22\linewidth}@{\hspace*{0.04\linewidth}}}p{0.22\linewidth}@{}}
        \setlength\tabcolsep{0pt}
        \@hangfrom{\textsf A.}#1  & \@hangfrom{\textsf B.}#2 & \@hangfrom{\textsf C.}#3 & \@hangfrom{\textsf D.}#4 \\
      \end{tabular}
    }%
  }%
  \unskip\unskip
}

\newcommand{\fivech}[5]{
  \par
  \settowidth{\choicelengtha}{A.#1}
  \settowidth{\choicelengthb}{B.#2}
  \settowidth{\choicelengthc}{C.#3}
  \settowidth{\choicelengthd}{D.#4}
  \settowidth{\choicelengthe}{E.#5}
  \ifthenelse{\lengthtest{\choicelengtha>\choicelengthb}}{\setlength{\maxlength}{\choicelengtha}}{\setlength{\maxlength}{\choicelengthb}}
  \ifthenelse{\lengthtest{\choicelengthc>\maxlength}}{\setlength{\maxlength}{\choicelengthc}}{}
  \ifthenelse{\lengthtest{\choicelengthd>\maxlength}}{\setlength{\maxlength}{\choicelengthd}}{}
  \ifthenelse{\lengthtest{\choicelengthe>\maxlength}}{\setlength{\maxlength}{\choicelengthe}}{}
  \ifthenelse{\lengthtest{\maxlength>0.5\linewidth}}
  {%
    \noindent%
    \begin{tabular}{@{}p{\linewidth}@{}}
      \setlength\tabcolsep{0pt}
      \@hangfrom{\textsf A.}#1 \\
      \@hangfrom{\textsf B.}#2 \\
      \@hangfrom{\textsf C.}#3 \\
      \@hangfrom{\textsf D.}#4 \\
      \@hangfrom{\textsf E.}#5 \\
    \end{tabular}
  }%
  {%
    \ifthenelse{\lengthtest{\maxlength>0.15\linewidth}}
    {%
      \noindent%
      \begin{tabular}{@{}p{0.48\linewidth}@{\hspace*{0.04\linewidth}}p{0.48\linewidth}@{}}
        \setlength\tabcolsep{0pt}
        \@hangfrom{\textsf A.}#1 & \@hangfrom{\textsf B.}#2 \\
        \@hangfrom{\textsf C.}#3 & \@hangfrom{\textsf D.}#4 \\
        \@hangfrom{\textsf E.}#5 \\
      \end{tabular}
    }%
    {%
      \noindent%
      \begin{tabular}{@{}*{5}{p{0.16\linewidth}@{\hspace*{0.04\linewidth}}}p{0.16\linewidth}@{}}
        \setlength\tabcolsep{0pt}
        \@hangfrom{\textsf A.}#1  & \@hangfrom{\textsf B.}#2 & \@hangfrom{\textsf C.}#3 & \@hangfrom{\textsf D.}#4 & \@hangfrom{\textsf E.}#5 \\
      \end{tabular}
    }%
  }%
  \unskip\unskip
}
\makeatother

% \renewcommand{\thesection}{\chinese{section}、}
% \renewcommand{\thesubsection}{\arabic{subsection}.}
% \renewcommand{\thesubsubsection}{(\arabic{subsubsection})}

% \makeatletter
% \renewcommand\@seccntformat[1]{%
%     {\csname the#1\endcsname}\hspace{0.1em}
% }
% \makeatother



\title{全国研究生招生考试 \\ 浙江中医药大学2020年攻读硕士学位研究生入学考试 \\ 《中医综合》试卷A代码615}
\author{开源:https://github.com/shujuecn/ZCMU615}
\date{}

\begin{document}

\maketitle

\section*{\large 一、A型题:1-80小题,每小题1.5分,共120分。在每小题给出的A、B、C、D四个选项中,请选出一项最符合题目要求的。}

\begin{enumerate}
      \item 金元四大家中,属补土派的医家是(\quad)
            \fourch{李杲}{朱丹溪}{刘完素}{张从正}
      \item 按五行理论,肺病及心的传变是(\quad)
            \fourch{相乘}{母病及子}{子病犯母}{相侮}
      \item “阴在内,阳之守也;阳在外,阴之使也”是指阴阳(\quad)
            \fourch{对立制约}{互根互用}{消长平衡}{交感互藏}
      \item “利小便即所以实大便”治法的理论依据是(\quad)
            \fourch{小肠泌别清浊}{脾运化水液}{肺宣发肃降}{肾主水液}
      \item 与胆汁产生密切相关的是(\quad)
            \fourch{先天之精}{水谷之精}{胆之余气}{肝之余气}
      \item 按分经诊断,上牙痛多在(\quad)
            \fourch{足阳明胃经}{手阳明大肠经}{手太阳小肠经}{足少阳胆经}
      \item 真实假虚证的治疗原则应是(\quad)
            \fourch{祛邪兼扶正}{扶正兼祛邪}{先祛邪后扶正}{单独祛邪}
      \item 具有“水火既济、精神互用”关系的脏腑是(\quad)
            \fourch{肺脾}{心脾}{心肾}{肺肾}
      \item 下列符合“用热远热”治则的是(\quad)
            \fourch{假寒慎用热药}{阴虚慎用热药}{夏季慎用热药}{假热慎用热药}
      \item 因虚致实,是指在正虚基础上又导致(\quad)
            \fourch{正气更加虚弱}{病理产物形成}{病情迁延不愈}{邪正斗争趋剧}
      \item 大怒主要损伤的脏是(\quad)
            \fourch{肝}{心}{脾}{肺}
      \item “夺血者无汗”的理论依据是(\quad)
            \fourch{精血同源}{精气互化}{气血同源}{津血同源}
      \item 后天各种因素使体质具有(\quad)
            \fourch{可变性}{稳定性}{全面性}{普遍性}
      \item 疾病复发的首要条件是(\quad)
            \fourch{饮食不慎}{邪未尽除}{情志失调}{劳逸失度}
      \item 下列各项,提出“独取寸口”诊脉法的是(\quad)
            \fourch{《黄帝内经》}{《难经》}{《脉经》}{《伤寒杂病论》}
      \item 循衣摸床属于(\quad)
            \fourch{得神}{少神}{失神}{假神}
      \item 面色苍白,伴四肢厥冷、冷汗淋漓,多见于(\quad)
            \fourch{阳虚}{阴虚}{亡阴}{亡阳}
      \item 手足蠕动的病机是(\quad)
            \fourch{热极生风}{血虚生风}{阴虚动风}{肝阳化风}
      \item 战汗后向愈的表现是(\quad)
            \fourch{精神萎靡}{身热不减}{烦躁不安}{脉静身凉}
      \item 根据五轮学说,风轮是指目的哪个部位(\quad)
            \fourch{瞳仁}{黑睛}{两眦}{白睛}
      \item 色红或青紫,点大成片,平摊于皮肤下,摸之不碍手,此为(\quad)
            \fourch{痈}{疹}{斑}{疽}
      \item 舌体瘦薄而色淡者,多属(\quad)
            \fourch{阴津亏虚}{热盛津伤}{气血两虚}{阳气亏虚}
      \item 灰黑苔辨寒热的关键是观察(\quad)
            \fourch{舌苔之厚薄}{舌苔之腐腻}{舌形之胖瘦}{舌苔之润燥}
      \item 口出酸腐气,伴脘腹胀满者,多属(\quad)
            \fourch{龋齿}{牙疳}{食积}{胃热}
      \item 突发耳鸣,声大如潮,按之不减,伴有口苦、胁肋灼痛,属于(\quad)
            \fourch{阴虚火旺}{肝胆火旺}{肝肾阴虚}{肾精亏损}
      \item 多食易饥,大便溏泄者,多属(\quad)
            \fourch{胃火亢盛}{胃阴不足}{胃强脾弱}{阴虚火旺}
      \item 脉“有神”主要是指(\quad)
            \fourch{不浮不沉,不大不小}{柔和有力,节律一致}{不快不慢,和缓有力}{尺脉有力,沉取不绝}
      \item 下列哪项对诊断心肾不交证最有意义(\quad)
            \fourch{心悸失眠,头晕目眩}{心悸怔忡,肢肿尿少}{心烦失眠,腰酸盗汗}{嗜睡神疲,心悸肢肿}
      \item 巴豆制霜的目的是(\quad)
            \fourch{增强药效}{降低毒性}{改变药性}{纯净药材}
      \item 确定归经学说的理论基础是(\quad)
            \fourch{阴阳学说}{药性理论}{脏腑经络学说}{药味理论}
      \item 羌活的性味是(\quad)
            \fourch{辛、甘,温}{辛、苦,温}{辛、涩,温}{辛、咸,温}
      \item 龙胆的归经是(\quad)
            \fourch{肺、肝}{脾、胆}{肝、肾}{肝、胆}
      \item 尤善治风湿顽痹的药物是(\quad)
            \fourch{独活}{乌梢蛇}{木瓜}{川乌}
      \item 既可燥湿温中,又能截疟除痰的药物是(\quad)
            \fourch{草豆蔻}{草果}{豆蔻}{砂仁}
      \item 下列药物中何药具有疏肝气之郁滞的功效(\quad)
            \fourch{附子}{丁香}{肉桂}{吴茱萸}
      \item 治疗脾胃气滞,脘腹胀痛及泻痢里急后重,宜选用(\quad)
            \fourch{陈皮}{枳壳}{大腹皮}{木香}
      \item 治疗血热夹瘀的出血证,宜选用(\quad)
            \fourch{地榆}{艾叶}{茜草}{仙鹤草}
      \item 治疗痰阻心窍所致的癫痫抽搐、惊风发狂者,宜选用(\quad)
            \fourch{琥珀}{磁石}{酸枣仁}{远志}
      \item 既治阳亢眩晕,又治瘰疬痰核,宜用(\quad)
            \fourch{石决明}{牡蛎}{刺蒺藜}{蜈蚣}
      \item 治疗阴虚有热,惊悸失眠者,宜选用(\quad)
            \fourch{南沙参}{北沙参}{百合}{石斛}
      \item 外用能收湿敛疮,且有解毒消肿之功的药物是(\quad)
            \fourch{海螵蛸}{五倍子}{赤石脂}{蒲公英}
      \item 外用杀虫主治疥疮,内服可助阳通便的药物是(\quad)
            \fourch{雄黄}{硫磺}{蛇床子}{土荆皮}
      \item 我国历史上第一部方论专著是(\quad)
            \fourch{《黄帝内经》}{《伤寒杂病论》}{《医方考》}{《普济方》}
      \item 银翘散中配伍荆芥穗、淡豆豉意在(\quad)
            \fourch{宣郁发表,疏风泄热}{解郁除烦,疏散风热}{辛散透邪,以助解表}{疏散风热,宣肺止咳}
      \item 胃肠燥热,津液不足,大便硬而小便数,舌质红,苔薄黄者,治宜选用(\quad)
            \fourch{济川煎}{增液汤}{五仁丸}{麻子仁丸}
      \item 凉膈散中用量最重的药物是(\quad)
            \fourch{连翘}{黄芩}{大黄}{芒硝}
      \item 患者,男,28岁。2天前出现发热,并伴有腹泻,下利臭秽,胸脘烦热,口干作渴,喘而汗出,舌红苔黄,脉数。辨证为(\quad)
            \fourch{腑热炽盛,热结旁流证}{邪热内陷,水热互结证}{表证未解,邪热入里证}{湿热壅滞,气血失调证}
      \item 真人养脏汤和四神丸组成中均含有的药物是(\quad)
            \fourch{罂粟壳}{木香}{补骨脂}{肉豆蔻}
      \item 天王补心丹和酸枣仁汤均具有的功用是(\quad)
            \fourch{养心安神}{补心安神}{镇心安神}{滋阴降火}
      \item 患者突然心腹猝痛,甚则昏厥,苔白,脉迟。治疗应首选(\quad)
            \fourch{紫金锭}{四逆汤}{四宝丹}{苏合香丸}
      \item 患者,女,55岁。哮喘咳嗽,痰多气急,咳吐黄稠痰,微恶风寒,舌苔黄腻,脉滑数。此病证的治法是(\quad)
            \fourch{辛凉疏表,清肺平喘}{解表散寒,温肺化饮}{宣降肺气,清热化痰}{降气平喘,祛痰止咳}
      \item 治疗热结下焦之血淋、尿血,首选方剂是(\quad)
            \fourch{导赤散}{八正散}{小蓟饮子}{十灰散}
      \item 头痛身热,干咳无痰,气逆而喘,咽喉干燥,胸满胁痛,心烦口渴,舌干无苔,脉虚大而数者,治宜选用(\quad)
            \fourch{桑杏汤}{益胃汤}{百合固金汤}{清燥救肺汤}
      \item 羌活胜湿汤的组成不含有的药物是(\quad)
            \fourch{川芎}{防风}{白芷}{藁本}
      \item 治疗湿温时疫,邪在气分,湿热并重的方剂是(\quad)
            \fourch{平胃散}{藿香正气散}{三仁汤}{甘露消毒丹}
      \item 患者眩晕,头痛,胸膈痞闷,恶心呕吐,舌苔白腻,脉弦滑。治宜选用的方剂是(\quad)
            \fourch{半夏白术天麻汤}{半夏泻心汤}{涤痰汤}{导痰汤}
      \item 论述燥邪伤肺为病而致咳嗽的证治,并创立了温润、凉润治咳之法的医家是(\quad)
            \fourch{张景岳}{吴鞠通}{叶天士}{喻嘉言}
      \item 肺阴亏虚咳嗽的主方是(\quad)
            \fourch{百合固金汤}{生脉饮}{麦门冬汤}{沙参麦冬汤}
      \item 患者咳吐大量脓痰,腥臭,伴有咳血,胸中烦痛,身热面赤,口渴,
            舌质红苔黄腻,脉滑数,治宜选用(\quad)
            \fourch{苇茎汤}{加味桔梗汤}{银翘散}{白虎汤}
      \item 患者心悸眩晕,胸闷痞满,下肢浮肿,形寒肢冷,小便短少,舌质淡胖苔白滑,脉弦滑,治宜选用(\quad)
            \fourch{苓桂术甘汤}{桂枝甘草龙骨牡蛎汤}{真武汤}{参附汤}
      \item 气滞心胸证胸痹的主症是(\quad)
            \fourch{心胸疼痛,如刺如绞,痛有定处}{心胸满闷,遇情志不遂时易诱发}{心胸隐痛,时作时休}{胸闷重而心痛微,痰多气短}
      \item 健忘的发生与下列哪些脏腑关系密切(\quad)
            \fourch{心、肺、肾}{心、脾、肾}{肝、脾、肾}{心、肝、脾}
      \item 《伤寒论》指出:“但满而不痛者,此为痞”,其治宜选用(\quad)
            \fourch{大陷胸汤}{小柴胡汤}{半夏泻心汤}{枳实消痞丸}
      \item 脾胃虚寒证胃痛患者,如泛吐酸水明显,可加用(\quad)
            \fourch{干姜、茯苓}{半夏、陈皮}{大枣、甘草}{吴茱萸、煅瓦楞子}
      \item 提出“调气则后重自除,行血则便脓自愈”治疗痢疾的医家是(\quad)
            \fourch{张仲景}{刘河间}{李东垣}{张景岳}
      \item 积聚的病位在(\quad)
            \fourch{肝脾}{脾肾}{肝胆}{脾胃}
      \item 太阴头痛常选用(\quad)
            \fourch{羌活}{柴胡}{细辛}{苍术}
      \item 患者证见腹部包块坚硬,剧痛,形体消瘦,神倦乏力,面色萎黄,舌质淡紫,舌光无苔,脉细数,治宜选用(\quad)
            \fourch{六磨汤}{八珍汤合化积丸}{柴胡疏肝散合失笑散}{膈下逐瘀汤}
      \item 《内经》提出“开鬼门”治疗水肿,属于八法中的(\quad)
            \fourch{吐法}{补法}{汗法}{消法}
      \item 患者血淋数月,小便涩滞不畅,尿色淡红如洗肉色,并见神疲乏力,面色少华,治宜选用(\quad)
            \fourch{无比山药丸}{补中益气汤}{春泽汤}{归脾汤}
      \item 虚劳辨证中,首先要辨别的要点是(\quad)
            \fourch{辨五脏气血阴阳亏虚}{辨本症与并发症}{辨原有疾病是否还继续存在}{辨有无因虚致实的表现}
      \item 瘀血腰痛,治宜选用(\quad)
            \fourch{身痛逐瘀汤}{独活寄生汤}{复元活血汤}{血府逐瘀汤}
      \item 患者头摇不止,肢麻震颤,头晕目眩,胸脘痞闷,口苦口黏,舌体胖大,有齿痕,舌质红苔黄腻,脉弦滑数,治宜选用(\quad)
            \fourch{地黄饮子}{黄连温胆汤}{导痰汤合羚角钩藤汤}{天麻钩藤饮合镇肝息风汤加减}
      \item 位于腕掌侧远端横纹上1寸,尺侧腕屈肌腱桡侧缘的穴位是(\quad)
            \fourch{神门}{阴郄}{通里}{灵道}
      \item 大肠的下合穴位于(\quad)
            \fourch{手阳明大肠经}{足阳明胃经}{手太阳小肠经}{任脉}
      \item 足阳明胃经起于(\quad)
            \fourch{目内眦}{目外眦}{鼻旁}{耳旁}
      \item 被称为“十二经脉之海”的是(\quad)
            \fourch{督脉}{任脉}{冲脉}{带脉}
      \item “连舌本,散舌下”的经脉是(\quad)
            \fourch{手少阴心经}{足少阴肾经}{足厥阴肝经}{足太阴脾经}
      \item 既是八会穴又是合穴的是(\quad)
            \fourch{曲池}{委中}{太渊}{阳陵泉}
      \item 下列腧穴中与志室穴相平的是(\quad)
            \fourch{脾俞}{肝俞}{肾俞}{大肠俞}
\end{enumerate}

\section*{\large 二、B型题:81-120小题:每小题1.5分,共60分。A、B、C、D是其下两道小题的备选项,请从中选择一项最符合题目要求的。每个选项可以被选择一次或两次。}

\begin{enumerate}[resume]
      \item[]
            \fourch{风}{燥}{火}{湿}
      \item 具有“黏滞”特性的病邪是(\quad)
      \item 具有“易伤肺”特性的病邪是(\quad)
            \fourch{主血脉}{主藏血}{主统血}{主朝百脉}
      \item 肝的主要功能是(\quad)
      \item 肺的主要功能是(\quad)
            \fourch{元气}{宗气}{营气}{卫气}
      \item 称为人体生命活动原动力的是(\quad)
      \item 具有温煦机体、防御外邪入侵功能的是(\quad)
            \fourch{喉中哮鸣}{腹部膨隆}{斜飞脉}{大便秘结}
      \item 上述各项,属于闻诊范畴的是(\quad)
      \item 上述各项,属于望诊范畴的是(\quad)
            \fourch{阳气不足}{营血亏虚}{血热亢盛}{虚阳外越}
      \item 面色淡白无华,唇舌色淡的临床的意义是(\quad)
      \item 面色皖白的临床意义是(\quad)
            \fourch{坐而仰首,胸胀气粗}{坐而喜俯,少气懒言}{仰卧伸足,掀去衣被}{蜷卧缩足,喜加衣被}
      \item 实热证多见的姿态为(\quad)
      \item 虚寒证多见的姿态为(\quad)
            \fourch{祛风湿,止痛,解表}{祛风湿,止痛,利水消肿}{祛风湿,利关节,解毒}{祛风湿,活血通络,清肺化痰}
      \item 独活的功效是(\quad)
      \item 羌活的功效是(\quad)
            \fourch{肺热咳喘}{寒湿带下}{鼻渊流涕}{水肿胀满}
      \item 白芷、细辛均治(\quad)
      \item 桑白皮、地骨皮均治(\quad)
            \fourch{尿血}{便血}{咯血}{吐血}
      \item 槐花长于治疗的出血症是(\quad)
      \item 小蓟长于治疗的出血症是(\quad)
            \fourch{发表解肌,温经通阳}{温通阳气,以助气化}{温经散寒,温通血脉}{温阳化气,解表散邪}
      \item 桂枝汤中桂枝的配伍意义是(\quad)
      \item 五苓散中桂枝的配伍意义是(\quad)
            \fourch{宜肺降气,祛痰平喘}{降气平喘,祛痰止咳}{宣降肺气,清热化痰}{降逆化痰,益气和胃}
      \item 苏子降气汤的功用是(\quad)
      \item 旋覆代赭汤的功用是(\quad)
            \fourch{羚角钩藤汤}{川芎茶调散}{四物汤}{天麻钩藤饮}
      \item 风邪外袭之头痛,治宜选用的方剂是(\quad)
      \item 肝阳偏亢,肝风上扰证之头痛,治宜选用的方剂是(\quad)
            \fourch{麻杏石甘汤}{桑白皮汤}{清金化痰汤}{泻白散}
      \item 痰热郁肺证喘证,治宜选用(\quad)
      \item 痰热郁肺证咳嗽,治宜选用(\quad)
            \fourch{归脾汤}{黄连阿胶汤}{安神定志汤}{酸枣仁汤}
      \item “少阴病,得之二三日以上,心中烦,不得卧”者,治宜选用(\quad)
      \item “虚劳虚烦,不得眠”者,治宜选用(\quad)
            \fourch{气滞血瘀}{胃失和降}{脾胃虚弱}{脾胃虚寒}
      \item 反胃的主要病机是(\quad)
      \item 呕吐的主要病机是(\quad)
            \fourch{头痛如裹}{头胀痛而眩}{头痛而空}{头痛而晕}
      \item 肝阳头痛的特点是(\quad)
      \item 风湿头痛的特点是(\quad)
            \fourch{三角窝前1/3的上部}{三角窝后1/3的上部}{三角窝前1/3的下部}{三角窝后1/3的下部}
      \item 耳穴“神门”位于(\quad)
      \item 耳穴“内生殖器”位于(\quad)
            \fourch{足少阳胆经}{手少阳三焦经}{足少阴肾经}{足太阳膀胱经}
      \item 根据十二经脉气血流注规律,手太阳小肠经的下一条经脉是(\quad)
      \item 根据十二经脉气血流注规律,手厥阴心包经的下一条经脉是(\quad)
            \fourch{督脉}{足太阳膀胱经}{足少阳胆经}{足阳明胃经}
      \item 额旁1线位于(\quad)
      \item 额旁2线位于(\quad)
            \fourch{疏波}{密波}{疏密波}{断续波}
      \item 常用于止痛、镇静、缓解肌肉和血管痉挛的电针波型是(\quad)
      \item 对横纹肌具有良好刺激收缩作用的电针波型是(\quad)
\end{enumerate}

\section*{\large 三、X型题:第121-180题,每小题2分,共120分。在每小题给出的A、B、C、D四个选项中,至少有两项是符合题目要求的。请选出所有符合题目要求的答案。多选或少选均不得分。}

\begin{enumerate}[resume]
      \item 下列属于瘀血致病的病症特点是(\quad)
            \fourch{倦怠乏力}{出血色暗}{刺痛拒按}{青紫肿胀}
      \item 下列各项中,依据五行相克规律确定的治法是(\quad)
            \fourch{滋水涵木法}{佐金平木法}{培土制水法}{补南泻北法}
      \item 下列与脾有关的是(\quad)
            \fourch{在窍为口}{在液为涎}{在体为筋}{其华在唇}
      \item 易伤津液的病邪是(\quad)
            \fourch{火邪}{燥邪}{暑邪}{风邪}
      \item 与人体呼吸有关的脏腑是(\quad)
            \fourch{肺}{肝}{脾}{肾}
      \item 下列脏腑中属于奇恒之腑的是(\quad)
            \fourch{髓}{胆}{脑}{女子胞}
      \item 同起于胞中的经脉是(\quad)
            \fourch{督脉}{带脉}{冲脉}{任脉}
      \item 脾与胃在生理上的关系表现在(\quad)
            \fourch{纳运相得}{燥湿相济}{精血互生}{升降相因}
      \item 与痰饮形成有关的脏腑是(\quad)
            \fourch{脾}{心}{肝}{肺}
      \item 从治法适用于(\quad)
            \fourch{气虚便秘}{肾虚尿闭}{食积腹泻}{瘀血所致崩漏}
      \item 下列属于青色主病的是(\quad)
            \fourch{寒证}{疼痛}{惊风}{血瘀}
      \item 下列属于黄疸病病机的有(\quad)
            \fourch{脾胃湿热}{肝胆湿热}{寒湿困脾}{肝火上炎}
      \item 短缩舌多见于(\quad)
            \fourch{寒凝}{痰阻}{血虚}{津伤}
      \item 下列各项,叙述正确的是(\quad)
            \fourch{头身困重为有湿}{结石阻塞为绞痛}{胁肋窜痛为肝郁}{发热皆因热邪所致}
      \item 下列各项,可出现心悸的有(\quad)
            \fourch{心肾不交}{心脉痹阻}{心阳亏虚}{心肝血虚}
      \item 痰与饮的共同之处有哪些(\quad)
            \fourch{均是由水液停聚而成}{均随气机升降无处不到}{形成均与肺脾肾有关}{都属于病理性产物}
      \item 口臭多见于(\quad)
            \fourch{口腔不洁}{龋齿}{消化不良}{便秘}
      \item 下列各项,不与壮热并见的是(\quad)
            \fourch{颧红盗汗}{烦渴引饮}{五心烦热}{舌红少苔}
      \item 太息可见于(\quad)
            \fourch{脾胃虚弱}{肝胃不和}{肝郁脾虚}{肝气郁滞}
      \item “癃闭"形成的原因有(\quad)
            \fourch{湿热下注}{瘀血阻滞}{砂石阻塞}{肾阳不足}
      \item 影响升降浮沉的主要因素是(\quad)
            \fourch{四气}{五味}{药物质地}{配伍}
      \item 能治麻疹不透、风疹瘙痒的药物是(\quad)
            \fourch{荆芥}{白芷}{蝉蜕}{牛蒡子}
      \item 可用于治疗虚热证的药是(\quad)
            \fourch{地骨皮}{知母}{牡丹皮}{黄柏}
      \item 茯苓的主治证为(\quad)
            \fourch{脾虚泄泻}{水肿}{痰饮}{惊悸}
      \item 既能活血,又能凉血的药物是(\quad)
            \fourch{茜草}{大黄}{姜黄}{郁金}
      \item 功能润肺止咳的药物有(\quad)
            \fourch{苦杏仁}{紫菀}{款冬花}{百部}
      \item 入汤剂宜后下的药物是(\quad)
            \fourch{桑叶}{肉桂}{钩藤}{砂仁}
      \item 僵蚕的主治病证是(\quad)
            \fourch{急惊风}{风热头痛}{风疹瘙痒}{瘰疬痰核}
      \item 龙骨和牡蛎的共同功效是(\quad)
            \fourch{平肝潜阳}{软坚散结}{收敛固涩}{制酸止痛}
      \item 南沙参与北沙参都具有的功效是(\quad)
            \fourch{补肺阴}{补胃阴}{补心阴}{补气}
      \item 体现“通因通用”反治法的方剂是(\quad)
            \fourch{芍药汤}{枳实导滞丸}{大承气汤}{保和丸}
      \item 下列方剂组成中含有半夏、黄芩的是(\quad)
            \fourch{半夏泻心汤}{达原饮}{小柴胡汤}{蒿芩清胆汤}
      \item 理气剂的适用证候包括(\quad)
            \fourch{脾胃气滞证}{肝气郁结证}{胃气上逆证}{肺气上逆证}
      \item 血府逐瘀汤和复元活血汤共有的药物是(\quad)
            \fourch{红花}{当归}{桃仁}{柴胡}
      \item 牛膝在镇肝熄风汤中的配伍意义是(\quad)
            \fourch{活血祛瘀}{补益肝肾}{利尿通淋}{引血下行}
      \item 均具有健脾利水功用的方剂是(\quad)
            \fourch{苓桂术甘汤}{实脾散}{防己黄芪汤}{五皮散}
      \item 下列属于固冲汤主治证候的是(\quad)
            \fourch{漏下不止}{月经过多}{血色深红}{舌淡脉细弱}
      \item 组成中含有大黄的方剂是(\quad)
            \fourch{桃核承气汤}{十灰散}{生化汤}{血府逐瘀汤}
      \item 可以治疗虚寒性腹痛的方剂有(\quad)
            \fourch{理中丸}{附子理中丸}{大建中汤}{小建中汤}
      \item 肾气丸的主治证包括(\quad)
            \fourch{虚劳腰痛}{痰饮}{消渴}{转胞}
      \item 《血证论》提出的治血法包括(\quad)
            \fourch{止血}{活血}{宁血}{补血}
      \item 治疗淋证的基本原则是(\quad)
            \fourch{实则清利}{虚则补益}{凉血止血}{通淋排石}
      \item 眩晕的病因与下列哪些因素有关(\quad)
            \fourch{肝阳上亢}{气血亏虚}{肾精不足}{痰湿中阻}
      \item 黄疸常并见于下列哪些病证(\quad)
            \fourch{胁痛}{胆胀}{鼓胀}{肝癌}
      \item 肺卫不固证自汗、盗汗,治宜选用(\quad)
            \fourch{桂枝加黄芪汤}{黄芪建中汤}{当归六黄汤}{玉屏风散}
      \item 腹痛的辨证要点是(\quad)
            \fourch{辨虚实}{辨性质}{辨急缓}{辨部位}
      \item 患者肌肉、关节刺痛数年,痛处固定不移,关节局部肿胀,按之稍硬,屈伸不利,舌质紫暗,苔白腻,脉弦涩。治宜选用(\quad)
            \fourch{祛风散寒}{除湿通络}{蠲痹通络}{化痰行瘀}
      \item 《难经》将泻分为五种,其中属于泄泻的是(\quad)
            \fourch{脾泄}{胃泄}{大瘕泄}{大肠泄}
      \item 正虚喘脱,治宜选用(\quad)
            \fourch{参附汤}{参蛤散}{生脉散}{黑锡丹}
      \item 癃闭的主症包括(\quad)
            \fourch{小便不利甚或小便闭塞}{每次尿量明显减少}{每日尿量明显减少}{可发现膀胱明显膨隆}
      \item 下列属于孕妇禁针的穴位是(\quad)
            \fourch{肩井}{合谷}{三阴交}{昆仑}
      \item 以下有关骨度折量寸描述正确的是(\quad)
            \fourch{前发际正中至后发际正中为9寸}{耳后两乳突之间为9寸}{两乳头之间为9寸}{腋前纹头至肘横纹为9寸}
      \item 以下有关经脉腧穴总数描述正确的是(\quad)
            \fourch{手少阴心经11}{手阳明大肠经20}{手少阳三焦经23}{足少阴肾经27}
      \item 以下属于艾灸法的是(\quad)
            \fourch{温和灸}{天灸}{太乙神针}{雷火神针}
      \item 有关顶颞前斜线主治描述正确的是(\quad)
            \fourch{上1/5治疗对侧下肢瘫痪}{上2/5治疗对侧下肢瘫痪}{中2/5治疗对侧上肢瘫痪}{下2/5治疗对侧头面部感觉异常}
      \item 以下属于同名经配穴法的是(\quad)
            \fourch{前额痛取合谷、内庭}{落枕取后溪、昆仑}{胃痛取手三里、足三里}{耳鸣取中渚、足临泣}
      \item 有关晕针处理描述正确的是(\quad)
            \fourch{立即停止针刺}{扶病人平卧,头部抬高}{松解衣带,注意保暖}{给饮温开水}
      \item 以下有关耳穴分布规律描述正确的是(\quad)
            \fourch{与面颊相应的穴位在耳垂}{与上肢相应的穴位在耳舟}{与躯干相应的穴位在对耳轮体部}{与内脏相应的穴位在耳轮脚}
      \item 以下经脉中循行至外眼角的是(\quad)
            \fourch{足少阳胆经}{足太阳膀胱经}{手少阳三焦经}{手太阳小肠经}
      \item 以下穴位中属于足少阳胆经的是(\quad)
            \fourch{肩髎}{率谷}{肩井}{听会}
\end{enumerate}

\end{document}
